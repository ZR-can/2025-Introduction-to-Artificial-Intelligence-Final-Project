% =============================================
% ==        sections/conclusion.tex          ==
% =============================================
\section{Conclusions and Future Work}\label{sec:conclusion}

Based on a comprehensive analysis of both theoretical experiments and real-world applications, MAPPO demonstrates clear advantages and wide applicability in the field of multi-agent learning. Test results show that in both simple and complex cooperative scenarios, MAPPO outperforms other methods across three key measures: Win-Rate, Allies-Dead-Ratio, and Enemies-Dead-Ratio. Its training framework effectively addresses information sharing and decision making coordination among multiple agents.

From a practical application perspective, MAPPO exhibits substantial potential in three major fields: robotic collaboration, autonomous driving, and military simulation. In industrial robotic arm cooperative assembly, it balances precision and efficiency; in autonomous vehicle platoon coordination, it ensures both safety and traffic fluency; in multi-branch military simulation exercises, it provides reliable support for tactical optimization. These application scenarios perfectly align with MAPPO's adaptability characteristics regarding collaboration complexity, environmental dynamics, and observation scope, further verifying its feasibility in transitioning from laboratory research to real-world applications.

Looking ahead, improvements should focus on key scenarios: enhancing how robotic arms process sensor information, improving self-driving cars' response to unexpected events, and refining military simulation tactics. By leveraging the method's adaptability across different equipment and scenarios, we can further strengthen its transition from theoretical framework to real-world implementation.