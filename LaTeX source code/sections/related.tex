% =============================================
% ==          sections/related.tex           ==
% =============================================
\section{Related Work}\label{sec:related}
\subsection{A Taxonomy of MARL Algorithms}
MARL algorithms can be categorized according to several key characteristics.
Certain classification criteria arise from the inherent properties of multi-agent systems, most notably the structure of the task. In contrast, other criteria, such as the explicit modeling or awareness of other agents, are unique to the context of multi-agent learning. 
One primary distinction separates algorithms based on their intended task: fully cooperative, fully competitive, or mixed stochastic games~\cite{busoniuComprehensiveSurveyMultiagent2008a}.
For a concise overview, Figure~\ref{fig:Breakdown of MARL algorithms} presents a breakdown of the surveyed algorithms based on these task categories.
It is worth noting that MAPPO, QMIX, and OQMIX discussed in this paper all fall into fully cooperative task type, while IPPO is a general-purpose one.
\begin{figure}[ht!] 
\centering

% 使用一个不可见的 tabular 来对齐顶部的两个表格
\begin{tabular}[t]{@{}l@{}}
    % 左边的 "Fully cooperative" 表格
    \begin{tabular}{|l|l|}
        \hline
        % 使用 \multicolumn 创建跨两列的居中标题
        \multicolumn{2}{|c|}{\textbf{Fully cooperative}} \\
        \hline
        \textbf{Static} & \textbf{Dynamic} \\
        \hline
        % 使用 \makecell 来实现单元格内换行
        % [l] 表示单元格内内容左对齐
        \makecell[l]{JAL \\ FMQ } & \makecell[l]{Team-Q \\ Distributed-Q  \\ OAL } \\
        \hline
    \end{tabular}
\end{tabular}
\hspace{0.2cm}
\begin{tabular}[t]{@{}l@{}}
    % 右边的 "Fully competitive" 表格
    \begin{tabular}{|l|}
        \hline
        \textbf{Fully competitive} \\
        \hline
        Minimax-Q  \\
        \hline
    \end{tabular}
\end{tabular}

\vspace{0.3cm} % 第一行表格和第二行表格之间的垂直间距

% --- 第二行:包含 "Mixed" 表格 ---
\begin{tabular}{|l|l|}
    \hline
    \multicolumn{2}{|c|}{\textbf{Mixed}} \\
    \hline
    \textbf{Static} & \textbf{Dynamic} \\
    \hline
    % 同样使用 \makecell 将所有条目放在一个单元格内并换行
    \makecell[l]{
        Fictitious Play  \\
        MetaStrategy\\
        IGA\\
        WoLF-IGA \\
        GIGA  \\
        GIGA-WoLF  \\
        AWESOME \\
        Hyper-Q 
    }
    &
    \makecell[l]{
        Single-agent RL  \\
        Nash-Q \\
        CE-Q  \\
        Asymmetric-Q\\
        NSCP  \\
        WoLF-PHC  \\
        PD-WoLF \\
        EXORL 
    } \\
    \hline
\end{tabular}
\caption{Breakdown of MARL algorithms by the type of task they address.Adapted from~\cite{busoniuComprehensiveSurveyMultiagent2008a},© 2008 IEEE.}\label{fig:Breakdown of MARL algorithms}
\end{figure}

\subsection{Value Function Decomposition for Cooperation}
A primary challenge in cooperative MARL is credit assignment: determining which agents contributed to the team's success or failure. Value Function Decomposition (VFD) methods address this by learning a global joint action-value function, $Q_{\text{tot}}$, as a composition of individual agent utility functions, $Q_i$. A key theoretical underpinning for many VFD methods is the Individual-Global-Max (IGM) principle, which ensures that a greedy selection of actions at the local level results in a globally optimal joint action~\cite{hongRethinkingIndividualGlobal2022}.

The pioneering work in this area, Value Decomposition Networks (VDN)~\cite{sunehagValueDecompositionNetworksCooperative2017}, proposed a simple summation:
\begin{equation}
    Q_{\text{tot}} = \sum_{i=1}^{n} Q_i.
    \label{eq:Qtot=Qi}
\end{equation}
While effective and compliant with the IGM principle, this additive approach restricts the representational capacity of the joint value function. Building upon this, QMIX~\cite{rashidQMIXMonotonicValue2018} introduces a mixing network that estimates $Q_{\text{tot}}$ from the individual $Q_i$ values. It enforces a monotonicity constraint,
\begin{equation}
    \frac{\partial Q_{\text{tot}}}{\partial Q_i} \ge 0,
    \label{eq:partial_derivative_condition}
\end{equation}
which is a less restrictive condition for satisfying the IGM principle, thereby allowing for a much richer class of joint value functions to be learned. Optimistically-Weighted QMIX (OW)~\cite{rashidWeightedQMIXExpanding2020}, an algorithm included in our study, further extends this by using an optimistic weighting scheme to improve credit assignment, especially in tasks with complex agent interactions.

\subsection{Actor-Critic Methods under Centralized Training}

As an alternative to VFD, the Centralized Training with Decentralized Execution (CTDE) paradigm has become a dominant force in MARL, particularly for actor-critic methods. The core philosophy of CTDE is to leverage global information during the training phase to mitigate the non-stationarity of the multi-agent learning problem, while ensuring that the resulting policies can be executed in a decentralized manner using only local observations.

In this framework, the critic is centralized and conditioned on the global state $s$ (or the joint observation-action history of all agents), providing a stable and accurate value estimate. This centralized critic then guides the updates for decentralized actors, which learn policies $\pi_i$ based on their local observation histories $\tau_i$. An early and influential off-policy example is MADDPG~\cite{loweMultiAgentActorCriticMixed2020}, which extends DDPG by augmenting each agent's critic with the policies of other agents.

Our primary algorithm of interest, MAPPO~\cite{yuSurprisingEffectivenessPPO2022}, adapts the on-policy Proximal Policy Optimization (PPO) algorithm to this CTDE framework. MAPPO inherits the stability and reliability of PPO, using a clipped surrogate objective to prevent destructive policy updates. Its use of a centralized critic provides a high-quality advantage estimate that is crucial for stable and efficient learning in complex cooperative scenarios, leading to its state-of-the-art performance across numerous benchmarks.

\subsection{Independent Learning}
Independent Proximal Policy Optimization (IPPO), a form of independent learning where each agent only estimates its local value function, can perform as well as or even better than state-of-the-art joint learning methods on the popular multi-agent benchmark suite SMAC, with almost no need for hyperparameter tuning~\cite{wittIndependentLearningAll2020}.
Despite its various theoretical shortcomings, IPPO, which possesses simplicity and scalability, can serve as a key comparison in our analysis to empirically quantify the benefits of the more complex coordination mechanisms adopted by the VFD and CTDE methods.
